\subsubsection{Beispiel - Lotto (Version Deutschland)}
Ziehe 6 aus 49 Kugeln (ohne Zusatzzahl). Was ist die Wahrscheinlichkeit für einen 6er, bzw 3er?\\
\enumstart
	\item Modell
	\enumstart
		\item Ein Elementarereignis ist das Ergebnis einer Ziehung, Reihenfolge ist egal $\implies$ typisches $\omega$ ist also eine Menge von 6 Zahlen.
		\item $\Omega = \{\omega = \{z_1, \mathellipsis, z_3\} | z_1, \mathellipsis, z_6 \in \{1, \mathellipsis, 49\}\}$
		\item $\F = 2^\Omega$
		\item $P =$ Gleichverteilung
		\item $\Omega = \binom{49}{6} = \frac{49*48*47*46*45*44}{6!}$
		\item $A_6 = \{\text{habe Sechser}\} = \{\text{gezogene Zahlen stimmen mit meinen, a priori gewählten, überein}\}$
		\item $|A_6| = 1 \implies P[A_6] = \frac{|A_6|}{|\Omega|} = \frac{1}{\binom{49}{6}} = 7.15*10^{-8}$
		\item $A_3 = \{$habe Dreier$\} = \{$3 der gezogenen 6 liegen in meinen 6, die restlichen 3 liegen nicht in meinen 6, liegen also in den restlichen 43$\}$
		\item $|A_3| = \binom{6}{3}\binom{43}{3} = \frac{6*5*4*43*42*41}{3!3!} = 246820$
	\enumend
\enumend

\subsection{Bedingte Wahrscheinlichkeiten}
\enumstart
	\item Definition
	\enumstart
		\item $(\Omega, \F, P)$ ein Wahrscheinlichkeits-Raum, $A, B$ Ereignisse mit $P[A] > 0$
		\item Bedingte Wahrscheinlichkeit von $B$ gegeben $A$ ist $P[B | A] := \frac{P[B \cap A]}{P[A]}$
		\item Beispiel - Würfelwurf
		\enumstart
			\item $A = \{$Gerade Augen zahl$\}$, $B = \{$Augenzahl  $<$ 4$\}$
			\item $\Omega = \{1, \mathellipsis, 6\}, \F = 2^\Omega, P =$ Gleichverteilung
			\item $A = \{2,4,6\}, B = \{1,2,3\} \implies A \cap B = \{2\} \implies P[B | A] = \frac{\frac{|A \cap B|}{|\Omega|}}{\frac{|A|}{|\Omega|}} = \frac{1}{3}$
		\enumend
	\enumend
	\item Bemerkung
	\enumstart
		\item Im Allgemeinen gilt: $P[B | A] \ne P[B]$
		\item Für fixiertes $A$ mit $P[A] > 0$ ist $P[\circ |A], B \mapsto P[B | A]$ wieder ein Wahrscheinlichkeitsmass auf $\F$ (Prüfe Axiome!)
		\item Ist $\Omega$ endlich oder abzählbar mit $\F = 2^\Omega$, so ist $P$ gegeben durch Gewichte $p_i = P[\{\omega_i\}]$.
		\item $P[\circ | A]$ hat Gewichte $p_i^A := P[\{\omega_i\} | A] = \begin{cases} 0 &\text{wenn }\omega_i \notin A\\\frac{p_i}{P[A]} &\text{wenn } \omega_i \in A \end{cases}$
		\item Bedingte Wahrscheinlichkeit ist nicht symmetrisch, für fixiertes $B$ ist $P[B | \circ], A \mapsto P[B | A]$ kein Wahrscheinlichkeitsmass
	\enumend
\enumend
Für beliebige Ereignisse $A, B$ gilt immer die Multiplikationregel $P[A \cap B] = P[A]P[B | A]$ (Klar aus der Definition, wenn man $\frac{0}{0}$ als irgendetwas definiert und x*0 = 0 setzt.)\\
Vergleiche Additionregel $P[B \cup A] = P[B] + P[A] - P[B \cap A]$
\subsubsection{Beispiel - Urne}
\enumstart
	\item Eine Urne enthält 3 rote und 1 blaue Kugel. Ziehe 2 Kugeln ohne zurücklegen. Mit welcher Wahrscheinlichkeit sind beide rot?
	\item Variante 1
	\enumstart
		\item $\Omega = \mathellipsis, \F = 2^\Omega, P =$ Gleichverteilung, $C = \{$2 rote$\}, |C| = ?$
	\enumend
	\item Variante 2
	\enumstart
		\item $R_i = \{i$-te gezogene Kugel ist rot$\}, i = 1, 2, C = R_1 \cap R_2.$ Dann ist $P[R_1] = \frac{3}{4}, P[R_2 | R_1] = \frac{2}{3}$, $P[C] = P[R_1 \cap R_2] = P[R_1]P[R_2 | R_1] = \frac{1}{2}$
	\enumend
\enumend

\subsection{Satz der totalen Wahrscheinlichkeiten}
Sei $A_1, \mathellipsis, A_n$ eine Zerlegung von $\Omega$ in Ereignisse, d.h. $A_1, \mathellipsis, A_n \in \F, \dot\bigcup_{i=1}^n+_i = \Omega$\\
Für jedes beliebige $B \in \F$ gilt dann $P[B] = \sum^n_{i=1}P[A_i]P[B|A_i]$\\
Dies gilt auch für abzählbare Zerlegungen, setze $n = \infty$\\
Beweis
\enumstart
	\item $B = B \cap \Omega = \dot\bigcup_{i=1}^nB\cap A_i \implies P[B] = \sum_{i=1}^nP[B \cap A_i] = \sum_{i=1}^nP[A_i]P[B | A_i]$
\enumend

\subsubsection{Beispiel}
\enumstart
	\item (Siehe Einführung) Eine Urne enthält gleich viele gewöhnliche und gezinkte Würfel. Letztere haben statt der 6 eine 7. Ziehe zufällig einen Würfel und würfle. Mit welcher Wahrscheinlichkeit ist die dann erhaltene Zahl gerade? $Z=\{$gezogener Würfel ist gezinkt$\}$, $G=\{$gewürfelte Zahl ist gerade$\}$
	\item $\Omega = Z \dot\cup Z^c$, $P[Z] = \frac{1}{2} = P[Z^c]$
	\item $P[G | Z^c] = \frac{1}{2}$, $P[G | Z] = \frac{1}{3}$
	\item Damit erhält man: $P[G] = P[Z]P[G | Z] + P[Z^c]P[G | Z^c] = \frac{1}{6} + \frac{1}{4} = \frac{5}{12} < \frac{1}{2}$
	\item Das ist intuitiv klar, die Wahrscheinlichkeit für eine gerade Zahl ist kleiner, als wenn man einen normalen Würfel hätte.
	\item Es ist auch eine Darstellung als Baumdiagramm möglich.
\enumend

\subsubsection{Beispiel}
\enumstart
	\item  Nehmen wir nun an, die gewürfelte Zahl ist ungerade. Wie gross ist dann die (bedingte) Wahrscheinlichkeit, dass der Würfel gezinkt ist? A priori, ohne Information, ist $P[Z] = \frac{1}{2}$
	\item Aber wir suchen nun $P[Z | G^c] = \frac{P[Z \cap G^c]}{P[G^c]} = \frac{P[Z]P[G^c | Z]}{1 - P[G]} = \frac{4}{7}$
\enumend

\subsubsection{Beispiel - Krankheitsdiagnose}
\enumstart
	\item Bekannt
	\enumstart
		\item In der ganzen Bevölkerung haben 0.1\% diese Krankheit.
		\item Von den kranken werden bei einer Untersuchung 90\% entdeckt
		\item Von den gesunden Personen werden bei der Untersuchung 99\% als gesund eingeschätzt
	\enumend
	\item Nun hat man eine Person aus der Bevölkerung und die Diagnose "krank".
	\item Wie gross ist die (bedingte) Wahrscheinlichkeit, dass die Person wirklich krank ist? 8.3\%
	\item $A := \{$Eine zufällige Person ist krank$\}$, $B := \{$Untersuchung einer zufälligen Person ergibt krank$\}$
	\item Dann $P[A] = 0.1\% = 0.001$, $P[A^c] = 1 - P[A] = 0.999$
	\item $P[B | A] = 90\% = 0.9$, $P[B^c | A] = 1 - P[B | A] = 0.1$
	\item $P[B^c | A^c] = 99\% = 0.99$, $P[B | A^c] = 1 - P[B^c | A^c] = 0.01$
\enumend

\subsection{Bemerkung}
\enumstart
	\item Ist $g: \R \rightarrow \R$ eine Abbildung, so ist die Zusammensetzung $Y = g \circ X$ wieder eine diskrete Zufallsvariable. Kurzschreibweise $Y = g(X)$
\enumend

\subsubsection{Beispiel}
\enumstart
	\item $Y = X^2, Y = e^X, Y = (X-a)^n \forall a \in \R, n \in \N$
\enumend

\subsection{Erwartungswerte}
\enumstart
	\item Grundidee: ZV $X$ ist eine Funktion ("mehrdimensionales Objekt") $\implies$ wir wollen eine Kennzahl finden ("eindimensionales Objekt"), die das durchschnittliche Verhalten von $X$ beschreibt
	\item Motivation: Würfelspiel\\
	\begin{tabular}{lllllll}
		Augenzahl & 1 & 2 & 3 & 4 & 5 & 6\\
		Gewinn (Rp) & 10 & 10 & 20 & 20 & 40 & 80\\
	\end{tabular}
	\enumstart
		\item Was ist ein "fairer" Einsatz für dieses Spiel? d.h. Durchschnittlicher Gewinn = Einsatz
		\item Annahme: $n$ Spielrunden, $n_i$ Mal erscheint $i$
		\item Dann ist die Aszahlung: $G(n) = 10n_1 + 10n_2 + 20n_3 + 20n_4 + 40n_5 + 80n_6$
		\item Durchschnittliche Auszahlung: $\frac{1}{n}G(n) \rightarrow$ Ist abhängig vom Ergebnis der Würfelwürfe
		\item Für grosse $n$ erwartet man: $\frac{n_i}{n} = p_i$ (frequentistische Interpretation)
		\item Also: Erwartungswert = $10p_1 + 10p_2 + 20p_3 + 20p_4 + 40p_5 + 80p_6$
		\item Fairer Würfel: $\forall i, p_i = \frac{1}{6} \implies$ Erwartungswert = $\frac{180}{6} = 30$Rp = fairer Einsatz
	\enumend
	\item Definition: Sei $X$ eine diskrete ZV mit Gewichtsfunktion $p_X$. $E[X] = \sum_{x_k \in \W(X)}x_kp_x(x_k)$ nennt man den Erwartungswert von $X$.
	\item Falls $\sum_{x_k \in \W(X)}|x_k|p_x(x_k) < \infty$, dann ist $E[X]$ der Erwartungswert von $X$, andernfalls existiert der Erwartungswert nicht.
\enumend

\subsubsection{Beispiel - Raclette}
\enumstart
	\item Roulette-Rad: $\Omega = \{00,0,1,\mathellipsis, 36\}, \F = 2^\Omega, P[\{\omega\}] = \frac{1}{|\Omega|} = \frac{1}{38} \forall \omega \in \Omega$ (Laplace Modell)
	\item $X(\omega) := $ Nettogewinn beim 1CHF auf ungerade = $\begin{cases}1 &\text{falls $\omega$ ungerade}\\0 &\text{falls $\omega$ gerade}\end{cases}$
	\item Gewichtsfunktion. $p_X(1) = \frac{18}{38}, p_X(-1) = \frac{20}{38} \implies E[X] = 1\frac{18}{38}+(-1)\frac{20}{38} = \frac{-1}{19}$ (unfair für Spieler)
\enumend

\subsection{Bemerkung}
\enumstart
	\item $E[X]$ kann auch als Summe über $\Omega$ geschrieben werden, sofern er existiert
	\item $E[X] = \sum_{\omega_i \in \Omega}X(\omega_i)P[\{\omega_i\}] = \sum_{\omega_i \in \Omega}p_iX(\omega_i)$
	\item Beweis: $E[X] $
		$= \sum_{x_k \in \W(X)}x_kP[X = x_k]$
		$= \sum_{x_k \in \W(X)}x_kP[\{\omega_i | X(\omega_i) = x_k\}]$
		$= \sum_{x_k \in \W(X)}(x_k \sum_{\omega_i \in \Omega, X(\omega_i)= x_k}P[\omega_i]) $
		$= \sum_{\omega_i \in \Omega}X(\omega_i)P[\{\omega_i\}]$
\enumend

\subsection{Erwartungswert beim Anwenden von Funktionen auf Zufallsvariablen}
\enumstart
	\item Sei $X$ eine diskrete ZV mit Gewichtsfunktion $p_X$ und sei $Y = g(X)$ für eine Funktion $g: \R \rightarrow \R$. Dann ist $E[Y] = E[g(X)] = \sum_{x_k \in \W(X)}g(x_k)p_X(x_k)$ sofern die Reihe absolut konvergiert, d.h. $\sum_{x_k \in \W(X)}|g(x_k)p_X(x_k)| < \infty$
	\item Beweis:
	\enumstart
		\item Für jedes $y_i \in \W(Y)$, setze $A_i := \{x_k \in \W(X), g(x_k) = y_i)\}$
		\item Dann ist $\{Y = y_i\} = \dot\bigcup_{x_k \in A}\{X = x_k\}$ (Urbild von $y_i$ unter $g$)
		\item Also $p_Y(y_i) = P[Y = y_i] = \sum_{x_k \in A_i}P[X = x_k] = \sum_{x_k \in A_i}p_X(x_k)$ und $\W(X) = \dot\bigcup_{y_i \in \W(Y)}\dot\bigcup_{x_k \in A_i}\{x_i\}$
		\item $\implies E[Y] = \sum_{y_i \in \W(Y)}y_ip_Y(y_i) = \sum_{y_i \in \W(Y)}y_i\sum_{x_k \in A_i}p_X(x_k)$
		\item $= \sum_{y_i \in \W(X)}\sum_{x_k \in A_i}g(x_k)p_X(x_k) = \sum_{x_k \in \W(X)}g(x_k)p_X(x_k)$
	\enumend
\enumend

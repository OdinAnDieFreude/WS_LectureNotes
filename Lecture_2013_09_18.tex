\section{Was ist Stochastik (Zufallslehre)?}
\enumstart
	\item Wahrscheinlichkeitstheorie
	\enumstart
		\item mathematische Modelle für zufällige Phänomene
	\enumend
	\item Statistik
	\enumstart
		\item Gegeben: Daten
		\item Suche ein wahrscheinlichkeitstheoretisches Modell, das die Daten gut erklärt
	\enumend
\enumend

\subsection{Beispiel}
\enumstart
	\item Eine Urne enthält verschiedene Würfel, zwei Sorten, je gleich viele. Bei einer Sorte ist die 6 durch eine 7 ersetzt. Ziehe blind einen Würfel, würfle damit, teile mit, ob der gewürfelte Wert gerade oder ungerade ist.
	\item Typische Fragen der Wahrscheinlichkeitstheorie
	\enumstart
		\item Wie wahrscheinlich ist es, dass eine gerade Zahl kommt?
		\item Ziehe $n$-mal einen Würfel, wie verhält sich fur $n \rightarrow \infty$ der Anteil von geraden Ergebnissen?
	\enumend
	\item Typische Fragen der Statistik
	\enumstart
		\item Sie haben die Ergebnisse von $n$ Würfen. Hat der Würfel eine 7 oder nicht?
		\item Wie zuverlässig ist ihre Antwort? Wette möglich? Oder brachen Sie mehr Daten?
	\enumend
\enumend

\section{Wahrscheinlichkeitstheorie}
\subsection{Wahrscheinlichkeiten}
Idee: Zufallsexperimente beschreiben Situationen, wo Ausgang/Ergebnisse nicht exakt vorhersagbar oder bekannt sind. Wir möchten eine mathematische Beschreibung und Quantifizierung.

\subsection{Beispiel-Situationen}
\enumstart
	\item Zufällige Experimente
	\item Sehr komplexe Phänomene (z.B. Aktienkurse)
\enumend

\subsubsection{Grundbegriffe}
Definitionen\\
\enumstart
	\item Ereignisraum / Grundraum: Menge $\Omega \neq \emptyset$ enthält Elemente $\omega \in \Omega$, das sind alle möglichen Ergebnisse des betrachteten Zufallsexperiments (Elementarereignisse, Ausgänge)
	\item Beispiele
	\enumstart
		\item Ein Würfelwurf: $\Omega = \{$1,2,3,4,5,6$\}$
		\item Zwei Münzwürfe: $\Omega = \{$KK, KZ, ZK, ZZ$\}$
		\item Jobs im Drucker: $\Omega = \N_0$
		\item Lebensdauer einer Glühbirne: $\Omega = \R^{+}_0$
		\item Aktienkurs: $\Omega = \{$alle Funktionen $\omega:\R^{+} \rightarrow \R^{+}\}$
	\enumend
	\item Die Potenzmenge von $\Omega$ (Menge aller Teilmengen von $\Omega$)
	\enumstart
		\item Notation: $\Pow(\Omega)$ oder $2^{\Omega}$
	\enumend
	\item Prinzipielles Ereigniss
	\enumstart
		\item Ist eine Teilmenge $A \subseteq \Omega$, also $A \in 2^{\Omega}$
	\enumend
	\item Klasse aller beobachtbaren Ereignisse $\F$
	\enumstart
		\item Ist eine Teilmenge von $2^{\Omega}$
		\item Für endlich oder abzählbare $\Omega$: nehme meistens $\F = 2^{\Omega}$
		\item Für überabzählbare $\Omega$ muss $\F$ eine echte Teilmenge von $2^{\Omega}$ sein, um mathematische Probleme zu vermeiden.
		\item $\F$ muss immer eine $\sigma$-Algebra sein
	\enumend
	\item $\sigma$-Algebra Axiome
	\enumstart
		\item $\Omega \in \F$
		\item $A \in \F \implies A^c \in \F$
		\item Sind $A_n \in \F$ für $n \in \N$, so ist auch $\bigcup_{n \in \N}A_n \in \F$
	\enumend
	\item Bemerkungen
	\enumstart
		\item Natürlich ist $2^{\Omega}$ eine $\sigma$-Algebra
		\item Für dasselbe Experiment kann es verschiedene Beschreibungen (d.h. Paare ($\Omega$,$\F$)) geben
	\enumend
\enumend

\subsubsection{Beispiele}
Normaler Würfel, ein Wurf; Als Ergebniss wird nur mitgeteilt, ob die Zahl gerade/ungerade ist.
\enumstart
	\item Modell 1
	\enumstart
		\item $\Omega_1 = \{$U,G$\}$
		\item $\F_1 = 2^{\Omega} = \{\Omega_1, \emptyset, \{$U$\}$,$\{$G$\}\}$
	\enumend
	\item Modell 2
	\enumstart
		\item $\Omega_2 = \{1,2,3,4,5,6\}$ aber beobachtbar sind nicht alle Teilmengen!
		\item $\F_2 = \{\Omega, \emptyset, \{1,3,5\},\{2,4,6\}\}$
	\enumend
	\item Grundannahme: bei unserem Zufallsexperiment kommt als Ausgang (bei jedem Versuch)  genau ein Elementarereignis $\omega \in \Omega$ heraus
	\item Sprechweise: für $A \in \F$ sagt man "$A$ tritt ein", falls das erhaltene $\omega$ in $A$ liegt. Damit kann man mit Mengenoperationen neue Ereignisse bilden.
	\item Mengenoperationen
	\enumstart
		\item Vereinigung $A \cup B$: "$A$ oder $B$ (oder beide) tritt ein" ($A,B \in \F \implies A \cup B \in \F$)
		\item Durchschnitt $A \cap B$: "$A$ und $B$ treten ein" ($A,B \in \F \implies A \cap B \in \F$)
		\item Komplement $A^c$: "$A$ tritt nicht ein" ($A \in \F \implies A^c \in \F$)
		\item Die $\sigma$-Algebra Axiome garantieren die Verwendbarkeit dieser Operationen
	\enumend
\enumend

\subsubsection{Wahrscheinlichkeit / Wahrscheinlichkeitsmass}
Definition: Abbildung $P: \F \rightarrow [0,1], A \mapsto P[A]$
\enumstart
	\item A0) $P[A] \ge 0 \forall A \in \F$
	\item A1) $P[\Omega] = 1$
	\item A2) $P[\dot\cup_{n = 1}^{\infty}A_n] = \sum_{n = 1}^{\infty}P[A_n]$
\enumend
Sprechweise: $P[A]$ ist die Wahrscheinlichkeit (WS) für das Ereignis $A$, WS dass $A$ eintritt, $\mathellipsis$\\
Einfache Rechenregeln:
\enumstart
	\item $P[A^c] = 1 - P[A]$
	\enumstart
		\item $1 = P[\Omega] = P[A] + P[A^c] \implies P[A^c] = 1 - P[A]$
	\enumend
	\item $P[\emptyset] = 0$
	\enumstart
		\item $1 = P[\Omega] = P[\Omega] + P[\emptyset] = 1 + P[\emptyset]\implies P[\emptyset] = 0$
	\enumend
	\item $A \subseteq B$ and $A,B \in \F \implies P[A] \le P[B]$
	\enumstart
		\item $B = B \cap \Omega = B \cap (A \dot\cup A^c) = (B \cap A) \dot\cup (B \cap A^c) = A \dot\cup (B \setminus A) \implies P[B] = P[A] + P[B \setminus A] \ge P[A]$
	\enumend
	\item Für beliebige $A,B \in \F$ gilt: $P[A \cup B] = P[A] + P[B] - P[A \cap B]$
\enumend
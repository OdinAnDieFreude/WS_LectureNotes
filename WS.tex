\documentclass[10pt]{article}
%Gummi|065|=)
\usepackage[utf8]{inputenc}
\usepackage{amsmath}
\usepackage{amssymb}
\usepackage{fullpage}
\usepackage{lscape}
\usepackage{graphicx}
\usepackage{stmaryrd}

\newcommand{\enumstart}{\begin{enumerate}}
\newcommand{\enumend}{\end{enumerate}}
\newcommand{\N}{\mathbb{N}}
\newcommand{\Z}{\mathbb{Z}}
\newcommand{\R}{\mathbb{R}}
\newcommand{\Pow}{\mathcal{P}}
\newcommand{\F}{\mathcal{F}}
\newcommand{\W}{\mathcal{W}}
\newcommand{\B}{\mathcal{B}}

\setcounter{tocdepth}{4}
\setcounter{secnumdepth}{4}

\title{\textbf{Wahrscheinlichkeit und Statistik Vorlesungsmitschriften}}
\author{Christoph Stillhard}
\date{}
\begin{document}

\maketitle
\pagenumbering{none}
\thispagestyle{empty}

\let\stdsection\section
\renewcommand\section{\newpage\stdsection}

\newpage
\pagenumbering{roman}
\setcounter{page}{1}
\tableofcontents

\newpage
\pagenumbering{arabic}
\setcounter{page}{1}

\section{Was ist Stochastik (Zufallslehre)?}
\enumstart
	\item Wahrscheinlichkeitstheorie
	\enumstart
		\item mathematische Modelle für zufällige Phänomene
	\enumend
	\item Statistik
	\enumstart
		\item Gegeben: Daten
		\item Suche ein wahrscheinlichkeitstheoretisches Modell, das die Daten gut erklärt
	\enumend
\enumend

\subsection{Beispiel}
\enumstart
	\item Eine Urne enthält verschiedene Würfel, zwei Sorten, je gleich viele. Bei einer Sorte ist die 6 durch eine 7 ersetzt. Ziehe blind einen Würfel, würfle damit, teile mit, ob der gewürfelte Wert gerade oder ungerade ist.
	\item Typische Fragen der Wahrscheinlichkeitstheorie
	\enumstart
		\item Wie wahrscheinlich ist es, dass eine gerade Zahl kommt?
		\item Ziehe $n$-mal einen Würfel, wie verhält sich fur $n \rightarrow \infty$ der Anteil von geraden Ergebnissen?
	\enumend
	\item Typische Fragen der Statistik
	\enumstart
		\item Sie haben die Ergebnisse von $n$ Würfen. Hat der Würfel eine 7 oder nicht?
		\item Wie zuverlässig ist ihre Antwort? Wette möglich? Oder brachen Sie mehr Daten?
	\enumend
\enumend

\section{Wahrscheinlichkeitstheorie}
\subsection{Wahrscheinlichkeiten}
Idee: Zufallsexperimente beschreiben Situationen, wo Ausgang/Ergebnisse nicht exakt vorhersagbar oder bekannt sind. Wir möchten eine mathematische Beschreibung und Quantifizierung.

\subsection{Beispiel-Situationen}
\enumstart
	\item Zufällige Experimente
	\item Sehr komplexe Phänomene (z.B. Aktienkurse)
\enumend

\subsubsection{Grundbegriffe}
Definitionen\\
\enumstart
	\item Ereignisraum / Grundraum: Menge $\Omega \neq \emptyset$ enthält Elemente $\omega \in \Omega$, das sind alle möglichen Ergebnisse des betrachteten Zufallsexperiments (Elementarereignisse, Ausgänge)
	\item Beispiele
	\enumstart
		\item Ein Würfelwurf: $\Omega = \{$1,2,3,4,5,6$\}$
		\item Zwei Münzwürfe: $\Omega = \{$KK, KZ, ZK, ZZ$\}$
		\item Jobs im Drucker: $\Omega = \N_0$
		\item Lebensdauer einer Glühbirne: $\Omega = \R^{+}_0$
		\item Aktienkurs: $\Omega = \{$alle Funktionen $\omega:\R^{+} \rightarrow \R^{+}\}$
	\enumend
	\item Die Potenzmenge von $\Omega$ (Menge aller Teilmengen von $\Omega$)
	\enumstart
		\item Notation: $\Pow(\Omega)$ oder $2^{\Omega}$
	\enumend
	\item Prinzipielles Ereigniss
	\enumstart
		\item Ist eine Teilmenge $A \subseteq \Omega$, also $A \in 2^{\Omega}$
	\enumend
	\item Klasse aller beobachtbaren Ereignisse $\F$
	\enumstart
		\item Ist eine Teilmenge von $2^{\Omega}$
		\item Für endlich oder abzählbare $\Omega$: nehme meistens $\F = 2^{\Omega}$
		\item Für überabzählbare $\Omega$ muss $\F$ eine echte Teilmenge von $2^{\Omega}$ sein, um mathematische Probleme zu vermeiden.
		\item $\F$ muss immer eine $\sigma$-Algebra sein
	\enumend
	\item $\sigma$-Algebra Axiome
	\enumstart
		\item $\Omega \in \F$
		\item $A \in \F \implies A^c \in \F$
		\item Sind $A_n \in \F$ für $n \in \N$, so ist auch $\bigcup_{n \in \N}A_n \in \F$
	\enumend
	\item Bemerkungen
	\enumstart
		\item Natürlich ist $2^{\Omega}$ eine $\sigma$-Algebra
		\item Für dasselbe Experiment kann es verschiedene Beschreibungen (d.h. Paare ($\Omega$,$\F$)) geben
	\enumend
\enumend

\subsubsection{Beispiele}
Normaler Würfel, ein Wurf; Als Ergebniss wird nur mitgeteilt, ob die Zahl gerade/ungerade ist.
\enumstart
	\item Modell 1
	\enumstart
		\item $\Omega_1 = \{$U,G$\}$
		\item $\F_1 = 2^{\Omega} = \{\Omega_1, \emptyset, \{$U$\}$,$\{$G$\}\}$
	\enumend
	\item Modell 2
	\enumstart
		\item $\Omega_2 = \{1,2,3,4,5,6\}$ aber beobachtbar sind nicht alle Teilmengen!
		\item $\F_2 = \{\Omega, \emptyset, \{1,3,5\},\{2,4,6\}\}$
	\enumend
	\item Grundannahme: bei unserem Zufallsexperiment kommt als Ausgang (bei jedem Versuch)  genau ein Elementarereignis $\omega \in \Omega$ heraus
	\item Sprechweise: für $A \in \F$ sagt man "$A$ tritt ein", falls das erhaltene $\omega$ in $A$ liegt. Damit kann man mit Mengenoperationen neue Ereignisse bilden.
	\item Mengenoperationen
	\enumstart
		\item Vereinigung $A \cup B$: "$A$ oder $B$ (oder beide) tritt ein" ($A,B \in \F \implies A \cup B \in \F$)
		\item Durchschnitt $A \cap B$: "$A$ und $B$ treten ein" ($A,B \in \F \implies A \cap B \in \F$)
		\item Komplement $A^c$: "$A$ tritt nicht ein" ($A \in \F \implies A^c \in \F$)
		\item Die $\sigma$-Algebra Axiome garantieren die Verwendbarkeit dieser Operationen
	\enumend
\enumend

\subsubsection{Wahrscheinlichkeit / Wahrscheinlichkeitsmass}
Definition: Abbildung $P: \F \rightarrow [0,1], A \mapsto P[A]$
\enumstart
	\item A0) $P[A] \ge 0 \forall A \in \F$
	\item A1) $P[\Omega] = 1$
	\item A2) $P[\dot\cup_{n = 1}^{\infty}A_n] = \sum_{n = 1}^{\infty}P[A_n]$
\enumend
Sprechweise: $P[A]$ ist die Wahrscheinlichkeit (WS) für das Ereignis $A$, WS dass $A$ eintritt, $\mathellipsis$\\
Einfache Rechenregeln:
\enumstart
	\item $P[A^c] = 1 - P[A]$
	\enumstart
		\item $1 = P[\Omega] = P[A] + P[A^c] \implies P[A^c] = 1 - P[A]$
	\enumend
	\item $P[\emptyset] = 0$
	\enumstart
		\item $1 = P[\Omega] = P[\Omega] + P[\emptyset] = 1 + P[\emptyset]\implies P[\emptyset] = 0$
	\enumend
	\item $A \subseteq B$ and $A,B \in \F \implies P[A] \le P[B]$
	\enumstart
		\item $B = B \cap \Omega = B \cap (A \dot\cup A^c) = (B \cap A) \dot\cup (B \cap A^c) = A \dot\cup (B \setminus A) \implies P[B] = P[A] + P[B \setminus A] \ge P[A]$
	\enumend
	\item Für beliebige $A,B \in \F$ gilt: $P[A \cup B] = P[A] + P[B] - P[A \cap B]$
\enumend
\subsection{Diskrete Wahrscheinlichkeits-Räume}
Oft ist $\Omega$ entweder endlich oder abzählbar; dann kann (und wird) man $\F = 2^\Omega$ wählen (d.h. alle prinzipiellen Ereignisse werden als beobachtbar angenommen).\\
Ferner kann man dann ein Wahrscheinlichkeitsmass $P$ festlegen durch die Werte $P[\{w_i\}] = p_i$ für alle $w_i \in \Omega$, d.h. man muss $P$ nur für einpunkt-Mengen spezifizieren, denn für $A \in \F = 2^\Omega$ ist dann $P[A] = P[\dot\bigcup_{w_i \in A}\{w_i\}] = \sum_{i \in [1, |A|]}P[\{w_i\}] = \sum_{i \in [1, |A|]}p_i$

\subsubsection{Spezifikation Einführung}
$\Omega$ ist endlich und alle $w_i \in \Omega$ haben gleiche Wahrscheinlichkeit.
\enumstart
	\item Wahrscheinlichkeit: Laplace-Modell
	\item $P$ = diskrete Gleichverteilung auf $\Omega$
	\item Klar: Ist $|\Omega| = N$, so ist dann $p_1 = p_2 = \mathellipsis = p_N = \frac{1}{N} = \frac{1}{|\Omega|}$
	\item Für $A \in \F = 2^\Omega$ ist dann $P[A] = \frac{|A|}{|\Omega|} = \frac{\text{Anzahl gültige Fälle}}{\text{Anzahl mögliche Fälle}}$
\enumend

\subsubsection{Beispiel - 2 Münzwürfe}
\enumstart
	\item $\Omega = \{KK, KZ, ZK, ZZ\}$
	\item $\F = 2^\Omega$
	\item $P =$ gleichverteilung, d.h. alle Sequenzen aus $K$ und $Z$  haben die gleiche Wahrscheinlichkeit
	\item Also: $p_i = \frac{1}{4}, i \in \{1,2,3,4\}$ und z.B. $P[\text{genau einmal Zahl}] = P[\{KZ, ZK\}] = \frac{2}{4} = \frac{1}{2}$
\enumend

\subsubsection{Beispiel - dreistellige Zahlen}
\enumstart
	\item Schreibe zufällig eine dreistellige Zahl hin. Wie wahrscheinlich ist es, dass sich dann Ziffern wiederholen?
	\item $\Omega = \{000, \mathellipsis, 999\}$
	\item $\F = 2^\Omega$
	\item $P =$ Gleichverteilung, d.h. "zufällig" wird interpretiert als "Alle Dreierserien haben gleiche Wahrscheinlichkeit"
	\item $A = \{\text{Geschriebene Zahl enthält wiederholende Ziffern}\}$
	\item Dann ist $|\Omega| = 1000$ aber $|A| = ?$
	\item Einfacher: $P[A] = 1 - P[A^c]$ und $|A^c| = |\{\text{Alle drei Ziffern sind verschieden}\}| = 10*9*8 = 720$
	\item $P[A^c] = 0,72 \implies P[A] = 0.28$
\enumend

\subsubsection{Beispiel - Geburtstagsproblem}
\enumstart
	\item Gegeben ist ein Raum mit $n$ Personen. Mit welcher Wahrscheinlichkeit hat man Pseudozwillinge d.h. Paare mit gleichem Geburtstag? Mit welcher Wahrscheinlichkeit hat 1 Person heute Geburtstag?
	\item Sei $t_i, i \in \{1, \mathellipsis, n\}$ der Geburtstag von Person $i$
	\item Ignoriere Schaltjahre (365 Tage pro Jahr)
	\item Elementarereignisse $\omega = (t_1, \mathellipsis, t_n) \implies \Omega = \{\text{Alle Folgen der Länge $n$ bestehend aus den Zahlen 1, $\mathellipsis$, 365}\}$
	\item $\F = 2^\Omega$
	\item $P =$ Gleichverteilung auf $\Omega$, d.h. alle möglichen Folgen werden als gleich wahrscheinlich angenommen.
	\item $|A| = \{\text{habe Pseudozwillinge}\}$
	\item Dann: $|\Omega| = 365^n$, aber $|A| = ?$
	\item $|A^c| = |\{\text{alle Geburtstage sind verschieden}\}| = 365*364*\mathellipsis*(365-(n-1))$
	\item $P[A^c] = \frac{|A|}{|\Omega|} = \frac{364}{365}*\mathellipsis*\frac{365-n+1}{365}$
	\item $B = \{\text{Jemand hat heute Geburtstag}\} \implies |B| = ?$
	\item $B^c = \{\text{Niemand hat heute Geburtstag}\} \implies |B^c| = 364^n$
	\item $P[B] = 1 - P[B^c] = 1 - (\frac{364}{365})^n$
	\item Ab welchem $n$ ist $P[B] \ge \alpha$?
	\item $1 - (\frac{364}{365})^n \ge \alpha \implies (\frac{364}{365})^n \le 1 - \alpha \implies n\log(\frac{364}{365}) \le \log(1 - \alpha) \implies n \ge \frac{\log(1 - \alpha)}{\log(\frac{364}{365})}$
\enumend

\subsubsection{Beispiel - Lotto (Version Deutschland)}
Ziehe 6 aus 49 Kugeln (ohne Zusatzzahl). Was ist die Wahrscheinlichkeit für einen 6er, bzw 3er?\\
\enumstart
	\item Modell
	\enumstart
		\item Ein Elementarereignis ist das Ergebnis einer Ziehung, Reihenfolge ist egal $\implies$ typisches $\omega$ ist also eine Menge von 6 Zahlen.
		\item $\Omega = \{\omega = \{z_1, \mathellipsis, z_3\} | z_1, \mathellipsis, z_6 \in \{1, \mathellipsis, 49\}\}$
		\item $\F = 2^\Omega$
		\item $P =$ Gleichverteilung
		\item $\Omega = \binom{49}{6} = \frac{49*48*47*46*45*44}{6!}$
		\item $A_6 = \{\text{habe Sechser}\} = \{\text{gezogene Zahlen stimmen mit meinen, a priori gewählten, überein}\}$
		\item $|A_6| = 1 \implies P[A_6] = \frac{|A_6|}{|\Omega|} = \frac{1}{\binom{49}{6}} = 7.15*10^{-8}$
		\item $A_3 = \{$habe Dreier$\} = \{$3 der gezogenen 6 liegen in meinen 6, die restlichen 3 liegen nicht in meinen 6, liegen also in den restlichen 43$\}$
		\item $|A_3| = \binom{6}{3}\binom{43}{3} = \frac{6*5*4*43*42*41}{3!3!} = 246820$
	\enumend
\enumend

\subsection{Bedingte Wahrscheinlichkeiten}
\enumstart
	\item Definition
	\enumstart
		\item $(\Omega, \F, P)$ ein Wahrscheinlichkeits-Raum, $A, B$ Ereignisse mit $P[A] > 0$
		\item Bedingte Wahrscheinlichkeit von $B$ gegeben $A$ ist $P[B | A] := \frac{P[B \cap A]}{P[A]}$
		\item Beispiel - Würfelwurf
		\enumstart
			\item $A = \{$Gerade Augen zahl$\}$, $B = \{$Augenzahl  $<$ 4$\}$
			\item $\Omega = \{1, \mathellipsis, 6\}, \F = 2^\Omega, P =$ Gleichverteilung
			\item $A = \{2,4,6\}, B = \{1,2,3\} \implies A \cap B = \{2\} \implies P[B | A] = \frac{\frac{|A \cap B|}{|\Omega|}}{\frac{|A|}{|\Omega|}} = \frac{1}{3}$
		\enumend
	\enumend
	\item Bemerkung
	\enumstart
		\item Im Allgemeinen gilt: $P[B | A] \ne P[B]$
		\item Für fixiertes $A$ mit $P[A] > 0$ ist $P[\circ |A], B \mapsto P[B | A]$ wieder ein Wahrscheinlichkeitsmass auf $\F$ (Prüfe Axiome!)
		\item Ist $\Omega$ endlich oder abzählbar mit $\F = 2^\Omega$, so ist $P$ gegeben durch Gewichte $p_i = P[\{\omega_i\}]$.
		\item $P[\circ | A]$ hat Gewichte $p_i^A := P[\{\omega_i\} | A] = \begin{cases} 0 &\text{wenn }\omega_i \notin A\\\frac{p_i}{P[A]} &\text{wenn } \omega_i \in A \end{cases}$
		\item Bedingte Wahrscheinlichkeit ist nicht symmetrisch, für fixiertes $B$ ist $P[B | \circ], A \mapsto P[B | A]$ kein Wahrscheinlichkeitsmass
	\enumend
\enumend
Für beliebige Ereignisse $A, B$ gilt immer die Multiplikationregel $P[A \cap B] = P[A]P[B | A]$ (Klar aus der Definition, wenn man $\frac{0}{0}$ als irgendetwas definiert und x*0 = 0 setzt.)\\
Vergleiche Additionregel $P[B \cup A] = P[B] + P[A] - P[B \cap A]$
\subsubsection{Beispiel - Urne}
\enumstart
	\item Eine Urne enthält 3 rote und 1 blaue Kugel. Ziehe 2 Kugeln ohne zurücklegen. Mit welcher Wahrscheinlichkeit sind beide rot?
	\item Variante 1
	\enumstart
		\item $\Omega = \mathellipsis, \F = 2^\Omega, P =$ Gleichverteilung, $C = \{$2 rote$\}, |C| = ?$
	\enumend
	\item Variante 2
	\enumstart
		\item $R_i = \{i$-te gezogene Kugel ist rot$\}, i = 1, 2, C = R_1 \cap R_2.$ Dann ist $P[R_1] = \frac{3}{4}, P[R_2 | R_1] = \frac{2}{3}$, $P[C] = P[R_1 \cap R_2] = P[R_1]P[R_2 | R_1] = \frac{1}{2}$
	\enumend
\enumend

\subsection{Satz der totalen Wahrscheinlichkeiten}
Sei $A_1, \mathellipsis, A_n$ eine Zerlegung von $\Omega$ in Ereignisse, d.h. $A_1, \mathellipsis, A_n \in \F, \dot\bigcup_{i=1}^n+_i = \Omega$\\
Für jedes beliebige $B \in \F$ gilt dann $P[B] = \sum^n_{i=1}P[A_i]P[B|A_i]$\\
Dies gilt auch für abzählbare Zerlegungen, setze $n = \infty$\\
Beweis
\enumstart
	\item $B = B \cap \Omega = \dot\bigcup_{i=1}^nB\cap A_i \implies P[B] = \sum_{i=1}^nP[B \cap A_i] = \sum_{i=1}^nP[A_i]P[B | A_i]$
\enumend

\subsubsection{Beispiel}
\enumstart
	\item (Siehe Einführung) Eine Urne enthält gleich viele gewöhnliche und gezinkte Würfel. Letztere haben statt der 6 eine 7. Ziehe zufällig einen Würfel und würfle. Mit welcher Wahrscheinlichkeit ist die dann erhaltene Zahl gerade? $Z=\{$gezogener Würfel ist gezinkt$\}$, $G=\{$gewürfelte Zahl ist gerade$\}$
	\item $\Omega = Z \dot\cup Z^c$, $P[Z] = \frac{1}{2} = P[Z^c]$
	\item $P[G | Z^c] = \frac{1}{2}$, $P[G | Z] = \frac{1}{3}$
	\item Damit erhält man: $P[G] = P[Z]P[G | Z] + P[Z^c]P[G | Z^c] = \frac{1}{6} + \frac{1}{4} = \frac{5}{12} < \frac{1}{2}$
	\item Das ist intuitiv klar, die Wahrscheinlichkeit für eine gerade Zahl ist kleiner, als wenn man einen normalen Würfel hätte.
	\item Es ist auch eine Darstellung als Baumdiagramm möglich.
\enumend

\subsubsection{Beispiel}
\enumstart
	\item  Nehmen wir nun an, die gewürfelte Zahl ist ungerade. Wie gross ist dann die (bedingte) Wahrscheinlichkeit, dass der Würfel gezinkt ist? A priori, ohne Information, ist $P[Z] = \frac{1}{2}$
	\item Aber wir suchen nun $P[Z | G^c] = \frac{P[Z \cap G^c]}{P[G^c]} = \frac{P[Z]P[G^c | Z]}{1 - P[G]} = \frac{4}{7}$
\enumend

\subsubsection{Beispiel - Krankheitsdiagnose}
\enumstart
	\item Bekannt
	\enumstart
		\item In der ganzen Bevölkerung haben 0.1\% diese Krankheit.
		\item Von den kranken werden bei einer Untersuchung 90\% entdeckt
		\item Von den gesunden Personen werden bei der Untersuchung 99\% als gesund eingeschätzt
	\enumend
	\item Nun hat man eine Person aus der Bevölkerung und die Diagnose "krank".
	\item Wie gross ist die (bedingte) Wahrscheinlichkeit, dass die Person wirklich krank ist? 8.3\%
	\item $A := \{$Eine zufällige Person ist krank$\}$, $B := \{$Untersuchung einer zufälligen Person ergibt krank$\}$
	\item Dann $P[A] = 0.1\% = 0.001$, $P[A^c] = 1 - P[A] = 0.999$
	\item $P[B | A] = 90\% = 0.9$, $P[B^c | A] = 1 - P[B | A] = 0.1$
	\item $P[B^c | A^c] = 99\% = 0.99$, $P[B | A^c] = 1 - P[B^c | A^c] = 0.01$
	\item Gesucht $P[A | B] = = \frac{P[A \cap B]}{P[B]} = \frac{P[A]P[B | A]}{P[A]P[B | A] + P[A^c]P[B | A^c]} = \frac{0.001*0.9}{0.001*0.9 + 0.999*0.01} = 0.0826$
	\item Bemerkung: Verbessert man Diagnostik von $P[B^c | A^c] = 0.99$ auf 99.9\%, dann steigt die Zuverlässigkeit (bedingte Wahrscheinlichkeit $P[A | B]$) von 8.26\% auf 47\%
\enumend

\subsection{Formel von Bayes}
\enumstart
	\item Bedingte Wahrscheinlichkeit von $B$ gegeben $A$, für $P[A] > 0$ ist $P[B | A] = \frac{P[B \cap A]}{P[A]}$
	\item Ist auch $P[B] > 0$, so ist $P[A | B] = \frac{P[A \cap B]}{P[B]} = \frac{P[B | A]P[A]}{P[B]} = \frac{P[B | A]P[A]}{P[B | A]P[A] + P[B | A^c]P[A^c]}$
	\item Allgemeiner: Ist $A_1, \mathellipsis, A_n$ eine Zerlegung von $\Omega $(d.h. $\dot\bigcup_{i=1}^nA_i = \Omega$) und $\forall i \in \{1, \mathellipsis, n\}.P[A_i] > 0$, so gilt die Formel von Bayes: $P[A_k | B] = \frac{P[B | A_k]P[A_k]}{\sum_{i=1}^nP[B | A_i]P[A_i]}$
\enumend

\subsubsection{Beispiel - Ziegenproblem}
\enumstart
	\item Drei Tore, zwei mit Ziegen, eine mit einem Preis dahinter
	\item $A := \{$bei der ersten gewählten Tür hat man den Preis erwischt$\}
$
	\item $G := \{$man gewinnt bei wechseln der Türe$\}$
	\item Mit bedingten Wahrscheinlichkeit
	\item Gesucht: $P[G] = ?$
	\enumstart
		\item Totale Wahrscheinlichkeit: $P[G] = P[G | A]P[A] + P[G | A^c]P[A^c] = 0*\frac{1}{3} + 1*\frac{2}{3} = \frac{2}{3}$
	\enumend
	\item Das gleiche ohne bedingte Wahrscheinlichkeiten
	\enumstart
		\item Habe zwei zufällige Elemente: stelle Preis hinter eine Tür und wähle Tür bei Schritt 1.
		\item $(i,k)$ für das Elementarereignis "Preis ist hinter Tür $i$, man wählt zu erst Tür $k$"
		\item $\Omega = \{(i,k) | i,k \in \{1,2,3\}\}, P =$ Gleichverteilung
		\item Für $i=k$ hat man bei Versuch 1 schon die richtige Tür, beim Wechsel verliert man. $\rightarrow$ 3 von 9 Fälle $\implies$ Wahrscheinlichkeit $\frac{1}{3}$
		\item Für $i \ne k$ hat man zu erst eine Falsche Tür, die zweite falcshe Tür wird geöffnet, also Gewinnt man bei einem Wechsel $\rightarrow$ 6 von 9 Fälle $\implies$ Wahrscheinlichkeit $\frac{2}{3}$
	\enumend
\enumend

\subsection{Unabhängigkeit}
\enumstart
	\item Definition: 2 Ereignisse $A,B$ heissen unabhängig (bezüglich $P$), falls $P[A \cap B] = P[A]P[B]$
	\item Bemerkung: Für $P[A] = 0$ oder $P[B] = 0$ sind $A,B$ immer unabhängig
	\item Für $P[A] > 0$ ist $P[B | A]$ wohldefiniert und dann gilt: $A,B$ unabhängig $\Leftrightarrow P[B | A] = P[B]$
	\item Analog für $P[B] > 0$ gilt: $B,A$ unabhängig $\Leftrightarrow P[A | B] = P[A]$
	\item Bemerkung:  Für mehr als 2 Ereignisse ist die Definition etwas Subtiler
\enumend

\subsubsection{Beispiel - Zwei Münzwürfe}
\enumstart
	\item Betrachte $K_i = \{$Kopf bei Wurf $i\}$
	\item Modell: $\Omega = \{$KK, KZ, ZK, ZZ$\}$, $\F = 2^\Omega$, $P =$ Gleichverteilung
	\item $|K_1| = |\{$KK, KZ$\}| = 2 \implies P[K_1] = \frac{1}{2}$
	\item $|K_2| = |\{$KK, ZK$\}| = 2 \implies P[K_2] = \frac{1}{2}$
	\item $|K_1 \cap K_2| = |\{$KK$\}| = 1 \implies P[K_1 \cap K_2] = \frac{1}{4} = P[K_1]P[K_2] \implies$ $K_1$ und $K_2$ sind unabhängig
\enumend

\subsubsection{Beispiel - Gezinkter Würfel (siehe Einführung)}
\enumstart
	\item $G_i = \{$Wurf $i$ gerade$ | i \in \{1,2\}\}$
	\item Sind $G_i$ unabhängig? Nein!
	\item Baumgrafik
\enumend


\subsubsection{Beispiel - Gezinkter Würfel 2}
\enumstart
	\item $Z = \{$gezogener Würfel ist gezinkt$\}, G_i = \{$gerade Zahl bei Wurf $i\}$
	\item $P[G_2 | G_1] = \frac{P[G_1 \cap G_2]}{P[G_1]} = \frac{13}{30} \ne \frac{5}{12} = P[G_2]$
	\item Sie sind nicht unabhängig, d.h. das Eintreten von $G_1$ ändert unsere Einschätzung für ein anschliessendes Eintreten von $G_2$
	\item Und: $P[G_1 | Z] \ne P[G_1], P[G_2 | Z] \ne P[G_2]$
	\item Intuitiv: $G_1$ und $G_2$ sind abhängig wegen der gemeinsamen Abhängigkeit von $Z$
	\enumstart
		\item TODO zeige oder widerlege Transitivität
	\enumend
\enumend

\subsubsection{Unabhängigkeit von mehreren Variablen}
\enumstart
	\item Eine beliebige Familie von Ereignisen $A_\lambda, \lambda \in \Lambda$ heisst unabhängig, falls für jede endliche Teilfamilie die Produktformel gilt. $\forall m \in \N, \forall \{\lambda_1, \mathellipsis, \lambda_m\} \subseteq \Lambda$ gilt $P[\bigcap^m_{i=1}A_{\lambda_i}] = \Pi^m_{i=1}P[A_{\lambda_i}]$
	\item Bemerkung: Für $n \ge 3$ Ereignisse macht es einen Unterschied, ob die Produktformel für alle endlichen Teilfamilien gilt(d.h. Unabhängigkeit), oder nur für alle Paare von Ereignissen(sogenannte paarweise Unabhängigkeit).
\enumend

\subsubsection{Beispiel Faire Münze, 2 Würfe}
\enumstart
	\item $A = \{$Kopf bei Wurf 1$\}$, $B = \{$Kopf bei Wurf 2$\}$, $C = \{$beide Würfe gleich$\}$
	\item $A,B$ und $C$ sind paarweise unabhängig
	\item $\Omega = \{$KK, KZ, ZK, ZZ$\}$, $\F = 2^\Omega$, $P =$ diskrete Gleichverteilung $\Rightarrow |\Omega| = 4$
	\item $A = \{$KK, KZ$\} \Rightarrow P[A] = \frac{1}{2}$
	\item $B = \{$KK, ZK$\} \Rightarrow P[B] = \frac{1}{2}$
	\item $C = \{$KK, ZZ$\} \Rightarrow P[C] = \frac{1}{2}$
	\item $A \cap B = \{KK\} \Rightarrow P[A \cap B] = \frac{1}{4} = P[A]P[B]$
	\item $A \cap C = \{KK\} \Rightarrow P[A \cap C] = \frac{1}{4} = P[A]P[C]$
	\item $B \cap C = \{KK\} \Rightarrow P[B \cap C] = \frac{1}{4} = P[B]P[C]$
	\item $A \cap B \cap C = \{KK\} \Rightarrow P[A \cap B \cap C] = \frac{1}{4} \ne \frac{1}{8} = P[A]P[B]P[C]$
	\item Es müssen also nicht nur Paare, sondern alle endlichen Teilmengen betrachtet werden
\enumend

\section{Diskrete Zufallsvariablen und diskrete Verteilungen}
\enumstart
	\item Bisher: beschreibe zufällige Experimente durch $(\Omega, \F, P)$
	\item Oft: ein Experiment ist (alternativ / einfacher) beschreibbar durch seine Ausgänge. Via eine Abbildung ist die besonders einfach, wenn $\Omega$ (oder etwas allgemeiner der Wertebereich der Abbildung) endlich oder abzählbar ist
	\item Im ganzen Kapitel: $\Omega \ne \emptyset$ ist endlich oder abzählbar, $\F = 2^\Omega$
	\item Damit ist $P$ beschreibbar durch $p_i = P[\{\omega_i\}] \forall i$
\enumend

\subsection{Grundbegriffe}
\enumstart
	\item Definition: Zufallsvariable (ZV) auf $(\Omega, \F)$ ist eine Abbildung $X: \Omega \rightarrow \R$
	\item Der Wertebereich $\W(X) \subseteq \R$ ist ebenfalls endlich oder abzählbar
	\item Verteilungsfunktion von $X$: Funktion $F_X: \R \rightarrow [0,1], t\mapsto F_X(t) := P[X \le t] = P[\{X \le t\}] = P[\{\omega \in \Omega | X(\omega \le t\}]$
	\item Gewichtsfunktion (diskrete Dichte): Funktion $p_X: \R \rightarrow [0,1], t \mapsto p_X(t) := P[X = t] = P[\{\omega \in \Omega | X(\omega) = t\}]$
	\item Klar für $t \notin \W(\Omega)$ ist $p_X(t) = 0$, also lebt $p_X$ auf $\Omega$
	\item Verteilung von $X$ genannt $\mu_X$: Wahrscheinlichkeitsmass auf $\R$ definiert durch $\mu_X(B) := P[X \in B]$ für $B \subseteq \R$. De facto braucht man nur $B \subseteq \W(X)$
	\item Bemerkung: Was passiert für $\Omega$ allgemein?
	\enumstart
		\item $p_X$ ist nutzlos, denn oft ist $p_X(t) = 0 \forall t$
		\item $F_X$ bleibt in der Definition unverändert, damit das aber Sinn macht, muss die Menge $\{X \le t\}$ in $\F$ sein, $\forall t \in \R$
		\item Eine Zufallsvariable ist eine messbare Abbildung, d.h. $X: \Omega \rightarrow \R$ mit $\{X \le t\} \in \F, \forall t \in \R$
		\item Verteilung $\mu_X$: sollte man nur auf $B \in \B(\R)$ definieren (d.h. nicht $B \in 2^\R$, sondern aus einer gewissen $\sigma$-Algebra auf $\R$)
	\enumend
\enumend

\end{document}